\documentclass["../Cours.tex"]{subfiles}

\begin{document}
\chapitre{Décomposition en facteurs premiers}


\partie{Nombres premiers}
\definition{On dit que $a$ est un diviseur de $b$ si le reste de la division euclidienne de $b$ par $a$ vaut 0.}

\exemple{3 est un diviseur de 15. Autrement dit, 15 est dans la table de 3.}

\definition{Un nombre premier est un entier $\neq 1$ ayant pour seuls diviseurs 1 et lui-même.}

\exemples{Les nombres premiers plus petits que 30 $\longrightarrow$ 2 ; 3 ; 5 ; 7 ; 11 ; 13 ; 17 ; 19 ; 23 ; 29}

\partie{Théorème fondamental de l'arithmétique}

\theoreme{}{Pour n'importe quel nombre entier $\supeg 2$, il existe une \underline{unique} façon de le décomposer en produits de facteurs premiers.}

\exemples{
    \begin{align*}
        219 &= 3 \times 73 \\[2ex]
        110 &= 10 \times 11 \\
        &= 2 \times 5 \times 11 \\[2ex]
        625 &= 5 \times 125 \\
        &= 5 \times 5 \times 25 \\
        &= 5 \times 5 \times 5 \times 5 \\
        &= 5^4
    \end{align*}
}


\clearpage
\EXERCICES
\begin{questions}
    \exercice Décomposer en facteurs premiers les nombres suivants
    \begin{multicols}{3}
        \questionX 95
        \questionX 105
        \questionX 344
        \questionX 72
        \questionX 100
        \questionX 87
        \questionX 66
        \questionX 3590
        \questionX 144
    \end{multicols}

    \exercice On veut offrir des biscuits au chocolat à des amis. Au magasin, on achète un paquet contenant 105 biscuits au chocolat blanc et 165 biscuits au chocolat au lait. On veut répartir de manière à ce que chaque ami reçoit le même nombre de biscuits au chocolat blanc et le même nombre de biscuits au chocolat au lait.
    \question À combien d'amis au maximum peut-on offrir de biscuits sans les casser ?

    \exercicetitre{Brevet - Métropole juin 2022}
    Une collectionneuse compte ses cartes Pokémon afin de les revendre. Elle possède 252 cartes de type \og feu \fg{} et 156 cartes de type \og terre \fg.

    \question 
        \subquestion Parmi les trois propositions suivantes, laquelle correspond à la décomposition en produit de facteurs premiers du nombre 252 :
        \begin{center}
        \begin{tabularx}{9cm}{|*{3}{>{\centering \arraybackslash}X|}}\hline
        Proposition 1 &Proposition 2& Proposition 3\\
         $2^2\times 9\times 7$ &$2\times 2\times 3\times 21$ &$2^2 \times 3^2\times  7$\\\hline
        \end{tabularx}
        \end{center}
        
        \subquestion Donner la décomposition en produit de facteurs premiers du nombre 156.

    \question Elle veut réaliser des paquets identiques, c'est-à-dire contenant chacun le même nombre de cartes << terre >> et le même nombre de cartes << feu >> en utilisant toutes ses cartes.
        \subquestion  Peut-elle faire 36 paquets ?
        \subquestion  Quel est le nombre maximum de paquets qu'elle peut réaliser ?
        \subquestion  Combien de cartes de chaque type contient alors chaque paquet ?
    
    \question  Elle choisit une carte au hasard parmi toutes ses cartes. On suppose les cartes  indiscernables au toucher. Calculer la probabilité que ce soit une carte de type << terre >>.

    \exercicetitre{Brevet - Centres étrangers 2022}
    Pour fêter les 25 ans de sa boutique, un chocolatier souhaite offrir aux premiers clients de la journée une boîte contenant des truffes au chocolat. Il a confectionné $300$ truffes: $125$ truffes parfumées au café et $175$ truffes enrobées de noix de coco. Il souhaite fabriquer ces boîtes de sorte que:

    \begin{itemize}
    \item[$\bullet~~$] Le nombre de truffes parfumées au café soit le même dans chaque boîte;
    \item[$\bullet~~$] Le nombre de truffes enrobées de noix de coco soit le même dans chaque boîte;
    \item[$\bullet~~$] Toutes les truffes soient utilisées.
    \end{itemize}

	\question Décomposer $125$ et $175$ en produit de facteurs premiers.
	\question En déduire la liste des diviseurs communs à $125$ et $175$.
	\question Quel nombre maximal de boîtes pourra-t-il réaliser ?
	\question Dans ce cas, combien y aura-t-il de truffes de chaque sorte dans chaque boîte ?

    \exercice Écrire un programme Scratch qui teste la primalité d'un nombre donné en entrée.

    \exercice Trouver 3 nombres premiers dont la somme vaut 2024.

    
\end{questions}

\end{document}