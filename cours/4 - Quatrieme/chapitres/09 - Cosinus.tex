\documentclass["../Cours.tex"]{subfiles}

\begin{document}
\chapitre{Cosinus}

\partie{Les côtés dans un triangle rectangle}

\definition{Le côté opposé à l'angle droit est appelé l'hypoténuse.}

\begin{center}
    \begin{tikzpicture}[scale=2]
        \draw (0,0) -- (0,1) -- (1.5,0) -- cycle;
        \draw (0,0) rectangle ++(0.2,0.2);
        \draw[-Latex] (1,1) node[right]{hypoténuse} arc(90:180:0.3);
    \end{tikzpicture}
\end{center}

\definition{Le côté adjacent à un angle aigüe est le côté qui forme l'angle qui n'est pas l'hypoténuse.}

\illustration{
    \begin{center}
        \begin{tikzpicture}
            \draw (0,0) -- (0,1.2) node[midway,left,vert]{adjacent} -- (2,0) -- cycle;
            \filldraw[green] (0,1.2) -- (0,1) arc(-90:-40:0.2) -- cycle;
            \filldraw[red] (0,0) rectangle (0.1,0.1);
            \draw[green,thick] (0,0) -- (0,1.2);
        \end{tikzpicture}
        \begin{tikzpicture}[xscale=-1,rotate=45]
            \draw (0,0) -- (0,1.2) node[midway,below right,vert]{adjacent} -- (2,0) -- cycle;
            \filldraw[green] (0,1.2) -- (0,1) arc(-90:-40:0.2) -- cycle;
            \filldraw[red] (0,0) rectangle (0.1,0.1);
            \draw[green,thick] (0,0) -- (0,1.2);
        \end{tikzpicture}
        \begin{tikzpicture}[rotate=78,xscale=-1,yshift=-1]
            \draw (0,0) -- (0,1.2) node[midway,above,vert]{adjacent} -- (2,0) -- cycle;
            \filldraw[green] (0,1.2) -- (0,1) arc(-90:-40:0.2) -- cycle;
            \filldraw[red] (0,0) rectangle (0.1,0.1);
            \draw[green,thick] (0,0) -- (0,1.2);
        \end{tikzpicture}
        \begin{tikzpicture}[rotate=194,xscale=1.5,yshift=-1]
            \draw (0,0) -- (0,1.2) -- (2,0) -- cycle node[midway,above,vert]{adjacent};
            \filldraw[green] (2,0) -- (1.8,0) arc(180:145:0.2) -- cycle;
            \filldraw[red] (0,0) rectangle (0.1,0.1);
            \draw[green,thick] (0,0) -- (2,0);
        \end{tikzpicture}
\end{center}
}

\definition{Le côté qui << ne touche pas >> l'angle aigüe choisi est le côté opposé à cet angle.}

\partie{Cosinus d'un angle}

\definition{Le cosinus d'un angle aigüe dans un triangle rectangle :
\begin{center}
    \begin{tikzpicture}
        \node[noir,below left] at (0,0) {$B$};
        \node[noir,left] at (0,3) {$A$};
        \node[noir,below] at (4,0) {$C$};
        \draw[noir] (0,0) -- (0,3) node[midway,left]{opp.} -- (4,0) node[midway,right]{hyp.} -- cycle node[midway,below]{adj.};
        \filldraw[green] (4,0) -- +(-0.3,0) arc(180:155:0.3) -- cycle;
        \node[noir] at (9,1.5) {\fbox{$\cos{(\widehat{ACB})} = \dfrac{\mbox{adjacent}}{\mbox{hypoténuse}} = \dfrac{BC}{AC} $}};
    \end{tikzpicture}
\end{center}
}

\exemple{
\begin{center}
    \begin{tikzpicture}
        \node[noir,below left] at (0,0) {$B$};
        \node[noir,left] at (0,3) {$A$};
        \node[noir,below] at (4,0) {$C$};
        \draw[noir] (0,0) -- (0,3) node[midway,left]{4} -- (4,0) node[midway,right]{5} -- cycle node[midway,below]{3};
        \filldraw[green] (4,0) -- +(-0.3,0) arc(180:155:0.3) -- cycle;
        \draw[Latex-,rouge] (-0.1,0.8) -- +(-0.8,0) node[left]{opp.};
        \draw[Latex-,rouge] (1,-0.1) -- +(0,-0.8) node[below]{adj.};
        \draw[Latex-,rouge] (1.7,2) -- +(0.5,0.5) node[above right]{hyp.};
    \end{tikzpicture}
\end{center}
\begin{align*}
    \cos{(\widehat{ACB})} &= \dfrac{\mbox{adjacent}}{\mbox{hypoténuse}} = \dfrac{BC}{AC} = \dfrac{4}{5} \\
    \widehat{ACB} &= \arccos{\left(\dfrac{4}{5}\right)} \approx \ang{36.86}
\end{align*}
}

\exemple{
    \begin{center}
        \begin{tikzpicture}
            \node[noir,below left] at (0,0) {$B$};
            \node[noir,left] at (0,3) {$A$};
            \node[noir,below] at (4,0) {$C$};
            \draw[noir] (0,0) -- (0,3) node[midway,left]{4} -- (4,0) node[midway,right]{5} -- cycle node[midway,below]{3};
            \filldraw[green] (4,0) -- +(-0.7,0) arc(180:145:0.7) -- cycle;
            \draw[Latex-,rouge] (-0.1,0.8) -- +(-0.8,0) node[left]{opp.};
            \draw[Latex-,rouge] (1,-0.1) -- +(0,-0.8) node[below]{adj.};
            \draw[Latex-,rouge] (1.7,2) -- +(0.5,0.5) node[above right]{hyp.};
        \end{tikzpicture}
    \end{center}
}

\clearpage
\EXERCICES
\begin{questions}
    \exercice Calculer la mesure de l'angle en vert dans chacun des angles ci-dessous.
    \begin{centre}
        \begin{tikzpicture}[scale=0.4]
            \draw (0,0) node[left]{$B$} -- (0,6) node[left]{$A$} node[rouge,midway,left]{6} -- (8,0) node[right]{$C$} node[rouge,midway,above right]{10} -- cycle node[rouge,midway,below]{8};
            \filldraw[noir] (0,0) rectangle +(0.7,0.7);
            \filldraw[green] (0,6) -- +(0,-0.6) arc(-90:-45:0.6);
        \end{tikzpicture}
        \begin{tikzpicture}[scale=0.4]
            \draw (0,0) node[left]{$E$} -- (0,5) node[left]{$D$} node[rouge,midway,left]{5} -- (12,0) node[right]{$F$} node[rouge,midway,above right]{13} -- cycle node[rouge,midway,below]{12};
            \filldraw[noir] (0,0) rectangle +(0.7,0.7);
            \filldraw[green] (0,5) -- +(0,-0.6) arc(-90:-35:0.6);
        \end{tikzpicture}
        \begin{tikzpicture}[scale=0.15]
            \draw (0,0) node[left]{$H$} -- (0,12) node[left]{$G$} node[rouge,midway,left]{12} -- (35,0) node[right]{$I$} node[rouge,midway,above right]{37} -- cycle node[rouge,midway,below]{35};
            \filldraw[noir] (0,0) rectangle +(2,2);
            \filldraw[green] (35,0) -- +(-4,0) arc(180:160:4);
        \end{tikzpicture}
    \end{centre}
    \exercice Calculer la mesure de l'angle en rouge ci-dessous.
    \begin{centre}
        \begin{tikzpicture}[scale=0.4]
            \draw (0,0) node[left]{$K$} -- (0,7) node[left]{$J$} node[rouge,midway,left]{7} -- (13.1,0) node[right]{$L$} node[rouge,midway,above right]{15} -- cycle;
            \filldraw[noir] (0,0) rectangle +(0.7,0.7);
            \filldraw[red] (13.1,0) -- +(-1.5,0) arc(180:150:1.5);
        \end{tikzpicture}
    \end{centre}

    \exercice Exprimer $\cos(\widehat{XYZ})$ et $\cos(\widehat{XZY})$ en fonction de $XY$, $YZ$, $XZ$.
    \begin{center}
        \begin{tikzpicture}
            \draw (0,0) node[left]{$X$} -- (3,0) node[right]{$Y$} -- (0,2) node[left]{$Z$} -- cycle;
            \filldraw[noir] (0,0) rectangle +(0.2,0.2);
        \end{tikzpicture}
    \end{center}

    \exercice Exprimer les cosinus des angles $\widehat{UWV}$, $\widehat{UVW}$, $\widehat{WSU}$, $\widehat{UWS}$
    \begin{center}
        \begin{tikzpicture}[rotate=30]
            \draw (0,0) node[left]{$U$} -- (3,0) node[below]{$V$} -- (0,2) node[left]{$W$} -- cycle;
            \draw (3,0) -- (4.5,0) node[below]{$S$} -- (0,2);
            \filldraw[noir] (0,0) rectangle +(0.2,0.2);
        \end{tikzpicture}
    \end{center}

    \exercice Compléter le tableau en arrondissant les valeurs au dixième.

    \begin{center}
    \begin{tabularx}{0.5\linewidth}{|l|X|X|X|X|}\hline 
    Angle & \ang{35} & & \ang{80} & \\\hline
    Cosinus &  & \num{0.3} &  & \num{0.98} \\\hline
    \end{tabularx}
    \end{center}

    \exercice $ABC$ est un triangle rectangle en $A$ tel que $AB=\qty{4}{\centi\metre}$ et $BC=\qty{8}{\centi\metre}$.
    \question Calculer la mesure de l'angle $\widehat{ABC}$ arrondie au degré.
    \question En déduire la mesure de l'angle $\widehat{ACB}$.

    \exercice $MNO$ est un triangle rectangle dont l'hypoténuse est $NO$. On donne $MO=\qty{4.3}{\centi\metre}$ et $\widehat{MON} = \ang{55}$.
    \question Faire une figure en vraie grandeur.
    \question Calculer la longueur $NO$ arrondie au millimètre.

    \exercice 
    \question Construire le triangle $ABC$ tel que $AC=\qty{5}{\centi\metre}$, $AB=\qty{6}{\centi\metre}$ et $\widehat{CAB} = \ang{40}$.
    \question Tracer la hauteur issue de $C$. On appelle $H$ le pied de la hauteur.
    \question Calculer la longueur $CH$ arrondie au millimètre.
    \question Calculer l'aire du triangle $ABC$.

    \exercice 
    \begin{center}
        \begin{tikzpicture}
            \draw (0,0) node[below]{$A$} -- ({3.3/tan(35)},0) node[right]{$D$} -- ({3.3/tan(35)},3.3) node[right]{$C$} node[midway,below,rotate=90]{\qty{3.3}{\centi\metre}} -- cycle;
            \draw (0,0) -- ({1.5*cos(125)},{1.5*sin(125)}) node[left]{$B$} -- ({3.3/tan(35)},3.3);
            \filldraw[noir,rotate=35] (0,0) rectangle (0.3,0.3);
            \filldraw[red] (0,0) -- (0.5,0) arc(0:35:0.5) node[midway,right]{\ang{35}} -- cycle;
            \filldraw[green] ({3.3/tan(35)},3.3) -- ($({3.3/tan(35)},3.3)!0.1!(0,0)$) arc(220:200:0.4) node[midway,above left]{\color{noir}\ang{17}} -- cycle; 
        \end{tikzpicture}
    \end{center}
    \question Calculer la mesure de l'angle $\widehat{ACD}$.
    \question Calculer la longueur $AC$ arrondie au millimètre.
    \question Calculer la longueur $BC$ arrondie au millimètre.

    \exercice $ABCD$ est un parallélogramme, tel que $AB=\qty{5}{\centi\metre}$, $BC=\qty{3}{\centi\metre}$ et $\widehat{DBC}=\ang{90}$.
    \question Faire un schéma de la figure.
    \question Calculer la mesure de l'angle $\widehat{BCD}$ arrondie au degré.
    \question Calculer la mesure de l'angle $\widehat{CDA}$ arrondie au degré.
\end{questions}

\clearpage
\CORRECTIONS
\begin{questions}
    \exercice 
    \question 
    \begin{align*}
        \cos{(\widehat{BAC})} &= \dfrac{\mbox{adj}}{\mbox{hyp}} = \dfrac{BA}{AC} = \dfrac{6}{10}\\
        \widehat{BAC} &= \arccos{\left(\dfrac{6}{10}\right)} \approx \ang{53.13}
    \end{align*}
    \question 
    \begin{align*}
        \cos{(\widehat{EDF})} &= \dfrac{\mbox{adj}}{\mbox{hyp}} = \dfrac{DE}{DF} = \dfrac{5}{13}\\
        \widehat{EDF} &= \arccos{\left(\dfrac{5}{13}\right)} \approx \ang{67.38}
    \end{align*}
    \question 
    \begin{align*}
        \cos{(\widehat{HIG})} &= \dfrac{\mbox{adj}}{\mbox{hyp}} = \dfrac{HI}{GI} = \dfrac{35}{37}\\
        \widehat{HIG} &= \arccos{\left(\dfrac{35}{37}\right)} \approx \ang{18.92}
    \end{align*}

    \exercice 
    Dans un premier temps, on va utiliser le théorème de Pythagore pour déterminer la longueur $KL$.
    \begin{enumerate}
        \item Le triangle $JKL$ est rectangle en $K$. Son hypoténuse est $[JL]$.
        \item D'après le théorème de Pythagore :
        \item 
        \begin{align*}
            JK^2+KL^2 &= JL^2 \\
            KL^2 &= JL^2 - JK^2 \\
            KL^2 &= 15^2-7^2 \\
            KL^2 &= 225-49 \\ 
            KL^2 &= 176 \\ 
            KL &= \sqrt{176} \approx \num{13.27} 
        \end{align*}
    \end{enumerate}

    Maintenant nous pouvons utiliser la formule du cosinus pour trouver l'angle.
    \begin{align*}
        \cos{(\widehat{JLK})} &= \dfrac{\mbox{adj}}{\mbox{hyp}} = \dfrac{KL}{JL} = \dfrac{\sqrt{176}}{15}\\
        \widehat{JLK} &= \arccos{\left(\dfrac{\sqrt{176}}{15}\right)} \approx \ang{27.82}
    \end{align*}


    \exercice 
    \begin{align*}
        \cos{\widehat{XYZ}} &= \dfrac{\mbox{adj}}{\mbox{hyp}} = \dfrac{XY}{YZ} \\
        \cos{\widehat{XZY}} &= \dfrac{\mbox{adj}}{\mbox{hyp}} = \dfrac{XZ}{YZ} \\
    \end{align*}

    \exercice 
    \begin{align*}
        \cos{\widehat{UWV}} &= \dfrac{\mbox{adj}}{\mbox{hyp}} = \dfrac{UW}{VW} \\
        \cos{\widehat{UVW}} &= \dfrac{\mbox{adj}}{\mbox{hyp}} = \dfrac{UV}{VW} \\
        \cos{\widehat{WSU}} &= \dfrac{\mbox{adj}}{\mbox{hyp}} = \dfrac{US}{WS} \\
        \cos{\widehat{UWS}} &= \dfrac{\mbox{adj}}{\mbox{hyp}} = \dfrac{WU}{WS} \\
    \end{align*}

    \exercice 
    \begin{center}
    \begin{tabularx}{0.9\linewidth}{|l|C|C|C|C|}\hline 
        Angle 
        & \ang{35} 
        & \makecell{\color{rouge} $\arccos(\num{0.3})$ \\ $\approx \ang{72.5}$ } 
        & \ang{80} 
        & \makecell{\color{rouge} $\arccos(\num{0.98})$ \\ $\approx \ang{11.5}$ } \\\hline
        Cosinus 
        & \makecell{\color{rouge} $\cos(\ang{35})$ \\ $\approx \num{0.8}$ } 
        & \num{0.3} 
        & \makecell{\color{rouge} $\cos(\ang{80})$ \\ $\approx \num{0.2}$ } 
        & \num{0.98} \\\hline
    \end{tabularx}
    \end{center}

    \exercice 
    \question
    \begin{align*}
        \cos{(\widehat{ABC})} &= \dfrac{\mbox{adj}}{\mbox{hyp}} = \dfrac{AB}{BC} = \dfrac{4}{8}\\
        \widehat{ABC} &= \arccos{\left(\dfrac{4}{8}\right)} = \ang{60}
    \end{align*}
    \question 
    Dans un triangle, la somme des mesures des angles d'un triangle vaut \ang{180}. On sait que $\widehat{BAC}=\ang{90}$ et $\widehat{ABC}=\ang{60}$. Donc :
    \[ \widehat{ACB} = 180 - (90 + 60) = 180 - 150 = \ang{30} \]

    \exercice 
    
\end{questions}

\end{document}