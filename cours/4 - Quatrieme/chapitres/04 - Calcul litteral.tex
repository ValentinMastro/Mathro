\documentclass["../Cours.tex"]{subfiles}

\begin{document}
\chapitre{Calcul littéral}

\partie{Expression littérale}

\souspartie{Écriture}
\definition{Une \emph{expression littérale} est un ensemble d'opérations contenant des nombres et des lettres.}
\definition{Dans une expression littérale, une lettre représente un nombre, cette lettre est appelée une \emph{variable}.}

\exemple{$3\times a + 2 \times b$ est une expression littérale contenant 2 variables : $a$ et $b$. Elle est composée de 3 opérations : 2 multiplications et 1 addition.}

\definition{Évaluer une expression signifie remplacer chacune des variables par une valeur numérique donnée.}

\exemple{Évaluer l'expression $3 \times a + 8$ en $a = 2$ signifie remplacer $a$ par 2, ce qui donne :  $3 \times 2 + 8 = 14$.}

\convention{%
Le symbole de multiplication est facultatif entre :
\begin{itemize}
    \item deux variables : $a\times b = ab$
    \item un nombre et une variable : $5 \times x = 5x$
    \item avec une parenthèses : $2 \times (x+3) = 2(x+3)$
\end{itemize}
}

\souspartie{Somme et produit}

\definition{Quand la dernière opération d'une expression littérale est une addition ou une soustraction, on dit que l'expression est une \emph{somme}.}

\definition{Quand la dernière opération d'une expression littérale est une multiplication, on dit que l'expression est un \emph{produit}.}

\begin{listedexemples}
    \item $2x+3$ est une somme
    \item $2(x+3)$ est un produit
\end{listedexemples}

\souspartie{Égalité d'expressions}

\definition{Deux expressions sont égales lorsque, \emph{quelques soient les valeurs choisies pour les variables}, on obtient le même résultat.}

\begin{listedexemples}
    \item Les expressions $a^2+3a$ et $a(a+3)$ sont égales.
    \item Les expressions $a^2$ et $2a+1$ ne sont pas égales. En effet, si $a=0$, $a^2=0$ mais $2a+1 = 1$.
\end{listedexemples}

\partie{Développer, factoriser et réduire}

\definition{Développer une expression littérale, c'est transformer un produit en une somme.}

\definition{Factoriser une expression littérale, c'est transformer une somme en produit.}

\souspartie{Distributivité}

\regle{Soient a, b et k, trois nombres : $k(a+b) = ka + kb$}

\souspartie{Double distributivité}

\regle{Soient a, b, c, d, quatre nombres : $(a+b)(c+d) = ac + ad + bc + bd$}

\clearpage
\begin{questions}
    \exercice Évaluer les expressions en $x=7$
    \question $2x+3$
    \question $9x-3$
    \question $x^2+12x-7$
    \question $(x+2)(x-8)$
    \question $(x+3)(3x+7)+2$

    \exercice Factoriser 
    \question $2\times 5 + 7 \times 5$
    \question $3xy + 2xz + 4xyz$
    \question $3x^2 + 6x$
    \question $12x - 30$
    \question $15x^2 + 18x$ 
    \question $27x^2 + 3$

    \exercice Développer et réduire
    \question $3(4x - 2)$ 
    \question $3x(4 + 8x) $
    \question $17x + 4x(5 - x) $
    \question $6(3 - 1,5x)-9x$
\end{questions}

\end{document}