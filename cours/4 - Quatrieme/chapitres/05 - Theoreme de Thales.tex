\documentclass["../Cours.tex"]{subfiles}

\begin{document}
\chapitre{Théorème de Thalès}

\partie{Énoncé}

\theoreme{de Thalès}{
Si : 
\begin{itemize}
    \item les points $A$, $B$ et $D$ sont alignés
    \item les points $A$, $C$ et $E$ sont alignés
    \item $(BC)\paral (DE)$
\end{itemize}
alors : 
$$\frac{AD}{AB} = \frac{AE}{AC} = \frac{DE}{BC}$$
}

\illustration{
\begin{center}
\begin{tikzpicture}
    \coordinate (A) at (0,0);
    \coordinate (D) at (3,1);
    \coordinate (E) at (3,-1);
    \coordinate (B) at ($(A)!0.65!(D)$);
    \coordinate (C) at ($(A)!0.65!(E)$);
    \draw (A) -- (D) -- (E) -- cycle (B) -- (C);
    \node[left] at (A) {$A$};
    \node[above left] at (B) {$B$};
    \node[below left] at (C) {$C$};
    \node[above right] at (D) {$D$};
    \node[below right] at (E) {$E$};
    \node at (1.5,-2.5) {Configuration n°1};
\end{tikzpicture}\hspace{2cm}
\begin{tikzpicture}
    \coordinate (A) at (0,0);
    \coordinate (D) at (3,1);
    \coordinate (E) at (3,-1);
    \coordinate (B) at ($(A)!-0.65!(D)$);
    \coordinate (C) at ($(A)!-0.65!(E)$);
    \draw (A) -- (D) -- (E) -- cycle (A) -- (B) -- (C) -- cycle;
    \node[above] at (A) {$A$};
    \node[left] at (B) {$B$};
    \node[left] at (C) {$C$};
    \node[right] at (D) {$D$};
    \node[right] at (E) {$E$};
    \node at (0.5,-2.5) {Configuration n°2 dite du << papillon >>};
\end{tikzpicture}
\end{center}
}

\begin{redaction}{
\begin{center}
\begin{tikzpicture}[scale=0.8]
    \draw (0,0) node[left] {$U$} -- (9,3) node[above left] {$X$};
    \draw (0,0) -- (9,-3) node[below left] {$Y$};
    \draw (4.5,1.5) node[above]{$V$} -- (4.5,-1.5) node[below]{$W$};
    \draw (8.4,2.8) -- (8.4,-2.8);
    \node[anchor=west] at (13,3) {$UV=4$};
    \node[anchor=west] at (13,2) {$VW=6$};
    \node[anchor=west] at (13,1) {$XY=24$};
    \node[anchor=west] at (13,0) {$UW=4,5$};
    \node[anchor=west] at (13,-1) {$(WV) \paral (XY)$};
    \node[anchor=west] at (13,-3) {$\Rightarrow$ Calculer $UX$};
\end{tikzpicture}
\end{center}
}

\underline{On sait que :}
\begin{itemize}
    \item Les points $U$, $V$ et $X$ sont alignés.
    \item Les points $U$, $W$ et $Y$ sont alignés.
    \item $(VW) \paral (XY)$
\end{itemize}

D'après le théorème de Thalès :
$$\dfrac{UX}{UV} = \dfrac{UY}{UW} = \dfrac{XY}{VW}$$

On remplace par les valeurs de l'énoncé :
$$\dfrac{UX}{4} = \dfrac{UY}{4,5} = \dfrac{24}{6}$$

On effectue un produit en croix :
\begin{center}
    \begin{tikzpicture}
        \node[anchor=west] at (0,0) {$\dfrac{UX}{4} = \dfrac{24}{6} \Rightarrow UX = \dfrac{4 \times 24}{6} = 16$};
        \draw[rouge,-latex] (0.92,-0.25) -- (1.6,0.3) -- (1.6,-0.25) -- (0.92,0.3);
    \end{tikzpicture}
\end{center}
\end{redaction}


\end{document}